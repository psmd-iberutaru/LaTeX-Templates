%% Beginning of file 'sample631.tex'
%%
%% Modified 2021 March
%%
%% This is a sample manuscript marked up using the
%% AASTeX v6.31 LaTeX 2e macros.
%%
%% AASTeX is now based on Alexey Vikhlinin's emulateapj.cls 
%% (Copyright 2000-2015).  See the classfile for details.

%%
%% using aastex version 6.3.1
\documentclass[linenumbers]{aastex631}

%% The default is a single spaced, 10 point font, single spaced article.
%% There are 5 other style options available via an optional argument. They
%% can be invoked like this:
%%
%% \documentclass[arguments]{aastex631}
%% 
%% where the layout options are:
%%
%%  twocolumn   : two text columns, 10 point font, single spaced article.
%%                This is the most compact and represent the final published
%%                derived PDF copy of the accepted manuscript from the publisher
%%  manuscript  : one text column, 12 point font, double spaced article.
%%  preprint    : one text column, 12 point font, single spaced article.  
%%  preprint2   : two text columns, 12 point font, single spaced article.
%%  modern      : a stylish, single text column, 12 point font, article with
%% 		  wider left and right margins. This uses the Daniel
%% 		  Foreman-Mackey and David Hogg design.
%%  RNAAS       : Supresses an abstract. Originally for RNAAS manuscripts 
%%                but now that abstracts are required this is obsolete for
%%                AAS Journals. Authors might need it for other reasons. DO NOT
%%                use \begin{abstract} and \end{abstract} with this style.
%%
%% Note that you can submit to the AAS Journals in any of these 6 styles.
%%
%% There are other optional arguments one can invoke to allow other stylistic
%% actions. The available options are:
%%
%%   astrosymb    : Loads Astrosymb font and define \astrocommands. 
%%   tighten      : Makes baselineskip slightly smaller, only works with 
%%                  the twocolumn substyle.
%%   times        : uses times font instead of the default
%%   linenumbers  : turn on lineno package.
%%   trackchanges : required to see the revision mark up and print its output
%%   longauthor   : Do not use the more compressed footnote style (default) for 
%%                  the author/collaboration/affiliations. Instead print all
%%                  affiliation information after each name. Creates a much 
%%                  longer author list but may be desirable for short 
%%                  author papers.
%% twocolappendix : make 2 column appendix.
%%   anonymous    : Do not show the authors, affiliations and acknowledgments 
%%                  for dual anonymous review.
%%
%% these can be used in any combination, e.g.
%%
%% \documentclass[twocolumn,linenumbers,trackchanges]{aastex631}
%%
%% AASTeX v6.* now includes \hyperref support. While we have built in specific
%% defaults into the classfile you can manually override them with the
%% \hypersetup command. For example,
%%
%% \hypersetup{linkcolor=red,citecolor=green,filecolor=cyan,urlcolor=magenta}
%%
%% will change the color of the internal links to red, the links to the
%% bibliography to green, the file links to cyan, and the external links to
%% magenta. Additional information on \hyperref options can be found here:
%% https://www.tug.org/applications/hyperref/manual.html#x1-40003
%%
%% Note that in v6.3 "bookmarks" has been changed to "true" in hyperref
%% to improve the accessibility of the compiled pdf file.
%%
%% If you want to create your own macros, you can do so
%% using \newcommand. Your macros should appear before
%% the \begin{document} command.
%%

\newcommand{\vdag}{(v)^\dagger}
\newcommand\aastex{AAS\TeX}
\newcommand\latex{La\TeX}

%% Custom packages and commands for Sparrow.
\RequirePackage{sparrow_packages}


%% Math symbols
%%%%%%%%%%%%%%%
\newcommand{\estimates}{\overset{\scriptscriptstyle\wedge}{=}}



%% Reintroduced the \received and \accepted commands from AASTeX v5.2
%\received{March 1, 2021}
%\revised{April 1, 2021}
%\accepted{\today}

%% Command to document which AAS Journal the manuscript was submitted to.
%% Adds "Submitted to " the argument.
%\submitjournal{PSJ}

%% For manuscript that include authors in collaborations, AASTeX v6.31
%% builds on the \collaboration command to allow greater freedom to 
%% keep the traditional author+affiliation information but only show
%% subsets. The \collaboration command now must appear AFTER the group
%% of authors in the collaboration and it takes TWO arguments. The last
%% is still the collaboration identifier. The text given in this
%% argument is what will be shown in the manuscript. The first argument
%% is the number of author above the \collaboration command to show with
%% the collaboration text. If there are authors that are not part of any
%% collaboration the \nocollaboration command is used. This command takes
%% one argument which is also the number of authors above to show. A
%% dashed line is shown to indicate no collaboration. This example manuscript
%% shows how these commands work to display specific set of authors 
%% on the front page.
%%
%% For manuscript without any need to use \collaboration the 
%% \AuthorCollaborationLimit command from v6.2 can still be used to 
%% show a subset of authors.
%
%\AuthorCollaborationLimit=2
%
%% will only show Schwarz & Muench on the front page of the manuscript
%% (assuming the \collaboration and \nocollaboration commands are
%% commented out).
%%
%% Note that all of the author will be shown in the published article.
%% This feature is meant to be used prior to acceptance to make the
%% front end of a long author article more manageable. Please do not use
%% this functionality for manuscripts with less than 20 authors. Conversely,
%% please do use this when the number of authors exceeds 40.
%%
%% Use \allauthors at the manuscript end to show the full author list.
%% This command should only be used with \AuthorCollaborationLimit is used.

%% The following command can be used to set the latex table counters.  It
%% is needed in this document because it uses a mix of latex tabular and
%% AASTeX deluxetables.  In general it should not be needed.
%\setcounter{table}{1}

%%%%%%%%%%%%%%%%%%%%%%%%%%%%%%%%%%%%%%%%%%%%%%%%%%%%%%%%%%%%%%%%%%%%%%%%%%%%%%%%
%%
%% The following section outlines numerous optional output that
%% can be displayed in the front matter or as running meta-data.
%%
%% If you wish, you may supply running head information, although
%% this information may be modified by the editorial offices.
\shorttitle{AASTeX v6.3.1 Sample article}
\shortauthors{Schwarz et al.}
%%
%% You can add a light gray and diagonal water-mark to the first page 
%% with this command:
%% \watermark{text}
%% where "text", e.g. DRAFT, is the text to appear.  If the text is 
%% long you can control the water-mark size with:
%% \setwatermarkfontsize{dimension}
%% where dimension is any recognized LaTeX dimension, e.g. pt, in, etc.
%%
%%%%%%%%%%%%%%%%%%%%%%%%%%%%%%%%%%%%%%%%%%%%%%%%%%%%%%%%%%%%%%%%%%%%%%%%%%%%%%%%
\graphicspath{{./}{figures/}}
%% This is the end of the preamble.  Indicate the beginning of the
%% manuscript itself with \begin{document}.

\begin{document}

\title{
  TITLE
}

%%
%% The \author command is the same as before except it now takes an optional
%% argument which is the 16 digit ORCID. The syntax is:
%% \author[xxxx-xxxx-xxxx-xxxx]{Author Name}
%%

%\correspondingauthor{August Muench}
%\email{greg.schwarz@aas.org, gus.muench@aas.org}

\author[0000-0002-1141-751X]{Kenji S. Emerson}
\affiliation{Institute of Astronomy Hilo, 640 North A`ohoku Place, Hilo, Hawaii 96720}



%% Mark off the abstract in the ``abstract'' environment. 
\begin{abstract}

ABSTRACT

\end{abstract}

%% Keywords should appear after the \end{abstract} command. 
%% The AAS Journals now uses Unified Astronomy Thesaurus concepts:
%% https://astrothesaurus.org
%% You will be asked to selected these concepts during the submission process
%% but this old "keyword" functionality is maintained in case authors want
%% to include these concepts in their preprints.
\keywords{Black holes (162)}

%% From the front matter, we move on to the body of the paper.
%% Sections are demarcated by \section and \subsection, respectively.
%% Observe the use of the LaTeX \label
%% command after the \subsection to give a symbolic KEY to the
%% subsection for cross-referencing in a \ref command.
%% You can use LaTeX's \ref and \label commands to keep track of
%% cross-references to sections, equations, tables, and figures.
%% That way, if you change the order of any elements, LaTeX will
%% automatically renumber them.
%%
%% We recommend that authors also use the natbib \citep
%% and \citet commands to identify citations.  The citations are
%% tied to the reference list via symbolic KEYs. The KEY corresponds
%% to the KEY in the \bibitem in the reference list below. 

\section{INTRODUCTION}

Hello world.


A meter is \qty{1}{m}.








%% IMPORTANT! The old "\acknowledgment" command has be depreciated. It was
%% not robust enough to handle our new dual anonymous review requirements and
%% thus been replaced with the acknowledgment environment. If you try to 
%% compile with \acknowledgment you will get an error print to the screen
%% and in the compiled pdf.
\begin{acknowledgments}
  ACKNOWLEDGMENTS
\end{acknowledgments}

%% To help institutions obtain information on the effectiveness of their 
%% telescopes the AAS Journals has created a group of keywords for telescope 
%% facilities.
%
%% Following the acknowledgments section, use the following syntax and the
%% \facility{} or \facilities{} macros to list the keywords of facilities used 
%% in the research for the paper.  Each keyword is check against the master 
%% list during copy editing.  Individual instruments can be provided in 
%% parentheses, after the keyword, but they are not verified.

\vspace{5mm}
\facilities{FACILITIES}

%% Similar to \facility{}, there is the optional \software command to allow 
%% authors a place to specify which programs were used during the creation of 
%% the manuscript. Authors should list each code and include either a
%% citation or url to the code inside ()s when available.

\software{SOFTWARE}

%% Appendix material should be preceded with a single \appendix command.
%% There should be a \section command for each appendix. Mark appendix
%% subsections with the same markup you use in the main body of the paper.

%% Each Appendix (indicated with \section) will be lettered A, B, C, etc.
%% The equation counter will reset when it encounters the \appendix
%% command and will number appendix equations (A1), (A2), etc. The
%% Figure and Table counter will not reset.

\appendix

\section{APPENDIX TITLE}

%% For this sample we use BibTeX plus aasjournals.bst to generate the
%% the bibliography. The sample631.bib file was populated from ADS. To
%% get the citations to show in the compiled file do the following:
%%
%% pdflatex sample631.tex
%% bibtext sample631
%% pdflatex sample631.tex
%% pdflatex sample631.tex

\bibliography{sample631}{}
\bibliographystyle{aasjournal}

%% This command is needed to show the entire author+affiliation list when
%% the collaboration and author truncation commands are used.  It has to
%% go at the end of the manuscript.
%\allauthors

%% Include this line if you are using the \added, \replaced, \deleted
%% commands to see a summary list of all changes at the end of the article.
%\listofchanges

\end{document}

% End of file `sample631.tex'.
